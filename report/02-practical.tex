\chapter{Практическая часть}

\section{Методика}
В качестве тестового набора данных использован ``Тест для французских булочек и чаю.zip'', содержащий изображения рукописных текстов на русском языке.

На каждой странице рукописного текста имеются метки позиционирования, пример которой представлен на рисунке \ref{img:poz_sign}, однако использовать их для увеличения точности распознавания не удалось.
Эти метки позиционирования система распознавания ошибочно интерпретировала как символы текста. Это снижало общую точность распознавания. Для устранения данной проблемы было принято решение предварительно удалять эти метки с изображений перед обработкой.

Стоит отметить, что основная задача этих меток - определение положения текста в пространстве - может быть выполнена всеми библиотеками, примененными в ходе данной работы, исключительно анализом текста, т.е. необходимости применения подобных меток в рамках данной работы нет.

\includeimage
    {poz_sign} % Имя файла без расширения (файл должен быть расположен в директории inc/img/)
    {f} % Обтекание (без обтекания)
    {h} % Положение рисунка (см. figure из пакета float)
    {0.5\textwidth} % Ширина рисунка
    {Метка позиционирования, распознанная как символ ©} % Подпись рисунка

Для каждого файла выполнялись следующие шаги:
\begin{enumerate}
    \item Предобработка изображения.
    \item Подача на вход каждого изображения, заранее обрезанного таким образом, чтобы в изображении остался только текст без меток позиционирования.
    \item Извлечение распознанного текста.
    \item Сравнение полученного текста с эталонным.
\end{enumerate}

\section{Результаты}

Распознавание текста, написанного низкоконтрастными чернилами, ожидаемо плохо выполнялось всеми тремя моделями. Результаты распознавания представлены на рисунках \ref{img:ocr10} и \ref{img:1_10_bboxes_easyocr}.

\includeimage
    {ocr10} % Имя файла без расширения (файл должен быть расположен в директории inc/img/)
    {f} % Обтекание (без обтекания)
    {h} % Положение рисунка (см. figure из пакета float)
    {0.8\textwidth} % Ширина рисунка
    {Результат разпознавания текста, полученный моделью \texttt{cyrillic\_PP-OCRv5\_mobile\_rec}} % Подпись рисунка


\begin{itemize}
    \item \textbf{PyTesseract} показал среднюю точность при работе с рукописным текстом. Библиотека часто ошибалась в распознавании похожих по начертанию символов кириллицы (например, ``и'' и ``н'', ``л'' и ``п'').
    \item \textbf{EasyOCR} точность распознавания была выше, чем у Tesseract, однако библиотека требовала больше времени на обработку. Показал лучшие результаты среди рассмотренных трех моделей.
    \item \textbf{PaddleOCR} с моделью \texttt{cyrillic\_PP-OCRv5\_mobile\_rec} показал худшие результаты среди рассмотренных трех моделей. Стоит отметить, что несмотря на то, что модель обучена на тексте с кириллицей, результатом распознавания отдельных символов являлись латинские буквы. Так буква Ш, часто распознавалась как W.
\end{itemize}

% \newpage

\includeimage
    {1_10_bboxes_easyocr} % Имя файла без расширения (файл должен быть расположен в директории inc/img/)
    {f} % Обтекание (без обтекания)
    {h} % Положение рисунка (см. figure из пакета float)
    {0.8\textwidth} % Ширина рисунка
    {Результат разпознавания текста, полученный библиотекой EasyOCR} % Подпись рисунка

Все изображения перед проведением процедуры распознавания текста были бинаризированы, как представлено на рисунке \ref{img:binary}.
\newpage
\includeimage
    {binary} % Имя файла без расширения (файл должен быть расположен в директории inc/img/)
    {f} % Обтекание (без обтекания)
    {h} % Положение рисунка (см. figure из пакета float)
    {0.5\textwidth} % Ширина рисунка
    {Бинаризированное изображение рукописного текста} % Подпись рисунка

Сравнительная таблица рассмотренных библиотек представлена в таблице \ref{tab:comparison}.

\begin{table}[h!]
\centering
\caption{Сравнительная таблица библиотек}
\label{tab:comparison}
\begin{tabular}{ >{\centering\arraybackslash}m{4cm} 
                >{\centering\arraybackslash}m{3.7cm} 
                >{\centering\arraybackslash}m{3.7cm} 
                >{\centering\arraybackslash}m{3.7cm} }
\toprule
\textbf{Критерий} & \textbf{PyTesseract} & \textbf{EasyOCR} & \textbf{PaddleOCR} \\
\midrule
Средняя символьная точность & 53\% & \textbf{55\%} & 45\% \\
\bottomrule
\end{tabular}
\end{table}

% \includelistingpretty
%     {out.txt} % Имя файла с расширением (файл должен быть расположен в директории inc/lst/)
%     {} % Язык программирования (необязательный аргумент)
%     {Пример работы с консольной программой} % Подпись листинга



\chapter*{ЗАКЛЮЧЕНИЕ}

В данной работе был проведён сравнительный анализ трёх библиотек для оптического распознавания текста: PyTesseract, EasyOCR и PaddleOCR. Тестирование выполнялось на наборе изображений рукописных текстов на русском языке.

Результаты эксперимента показали, что библиотека EasyOCR показала наилучшую точность распознавания (55\%), опередив PyTesseract (53\%) и PaddleOCR (45\%). PaddleOCR, несмотря на специализированную кириллическую модель, демонстрировал характерные ошибки, такие как замена кириллических символов на латинские.

Основные выводы исследования:
\begin{enumerate}
    \item Все рассмотренные библиотеки испытывают значительные трудности с распознаванием рукописного текста, что указывает на сложность самой задачи.
    \item Качество распознавания критически зависит от предобработки изображений. Удаление посторонних элементов и бинаризация являются необходимыми этапами.
    \item Для задач локального распознавания русской рукописи библиотека EasyOCR является наиболее предпочтительным выбором из рассмотренных.
\end{enumerate}

Как правило, обучение моделей машинного обучения для распознавания рукописного текста проводят на словах и предложениях, а не просто на совокупности букв. Этим можно объяснить низкую точность распознавания, полученную в ходе работы.

