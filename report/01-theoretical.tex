\chapter*{ВВЕДЕНИЕ}
Целью данной лабораторной работы является сравнительный анализ трёх локально разворачиваемых библиотек оптического распознавания символов (OCR), поддерживающих русский язык, при работе с рукописным текстом. Исследование проводилось на наборе данных ``Тест для французских булочек и чаю'', содержащем изображения рукописных текстов. Актуальность работы обусловлена растущей потребностью в автоматизации извлечения информации из рукописных источников в различных областях. В качестве объектов сравнения выбраны библиотеки: PyTesseract, EasyOCR и PaddleOCR с моделью \texttt{cyrillic\_PP-OCRv5\_mobile\_rec}.

\chapter{Теоретическая часть}
Оптическое распознавание символов (OCR) — это технология преобразования изображений печатного или рукописного текста в машиночитаемый текст. Современные системы OCR используют глубокое обучение, в частности, сверточные нейронные сети для детекции текста и рекуррентные нейронные сети для его распознавания.

\begin{itemize}
    \item \textbf{TesseractOCR} -- открытая библиотека, изначально разработанная Hewlett-Packard. В настоящее время поддерживается Google. Использует LSTM-сети. Позволяет работать с множеством языков, включая русский.
    \item \textbf{EasyOCR} -- библиотека на основе PyTorch, использующая архитектуры CRAFT для детекции текста и CRNN (CNN + BiLSTM) с механизмом внимания для распознавания. Поддерживает более 80 языков, включая русский.
    \item \textbf{PaddleOCR} -- фреймворк от Baidu, предлагающий современные предобученные модели. Модель \texttt{cyrillic\_PP-OCRv5\_mobile\_rec}, используемая в рамках работы, оптимизирована для распознавания кириллицы.
\end{itemize}

Основная метрика для оценки качества OCR -- точность распознавания -- отношение правильно распознанных символов к общему количеству символов в эталонном тексте.



% \includeimage
%     {1_1_ocr_res_img} % Имя файла без расширения (файл должен быть расположен в директории inc/img/)
%     {f} % Обтекание (без обтекания)
%     {h} % Положение рисунка (см. figure из пакета float)
%     {0.8\textwidth} % Ширина рисунка
%     {Схема предметной области} % Подпись рисунка
